% !TeX program = lualatex


\documentclass[11pt, a4paper]{article}

\usepackage{fontspec}
\usepackage{polyglossia}

% \setdefaultlanguage{russian}
\setmainlanguage{russian}
\setotherlanguages{english}

% download "Linux Libertine" OTF-fonts:
% http://www.linuxlibertine.org/index.php?id=91&L=1
\setmainfont{Linux Libertine O} % or Helvetica, Arial, Cambria
% why do we need \newfontfamily:
% http://tex.stackexchange.com/questions/91507/
\newfontfamily{\cyrillicfonttt}{Linux Libertine O}
\newfontfamily{\cyrillicfont}{Linux Libertine O}
\newfontfamily{\cyrillicfontsf}{Linux Libertine O}
 
\usepackage{etoolbox} % to use ifdef, must be after babel

\newtoggle{excerpt}
\togglefalse{excerpt}  % помечаем, что это не отрывок, а далее в тексте может использовать
 

\usepackage[paper=a4paper,
top=15mm,
bottom=13.5mm,
left=13mm, right=13mm, includefoot]{geometry}

\usepackage{etex} % расширение классического tex
% в частности позволяет подгружать гораздо больше пакетов, чем мы и займёмся далее




\usepackage{makeidx} % для создания предметных указателей
\usepackage{verbatim} % для многострочных комментариев
%\usepackage[pdftex]{graphicx} % для вставки графики
% omit pdftex option if not using pdflatex

\usepackage{comment} % для команды excludecomment


%\usepackage{dsfont} % шрифт для единички с двойной палочкой (для индикатора события)
\usepackage{bbm} % шрифт - двойные буквы


\usepackage[usenames, dvipsnames, svgnames, table, rgb]{xcolor}

\usepackage{colortbl}


% пакет для тестов:
% \usepackage[box, % запрет на перенос вопросов
% nopage, % убираем колонтитулы страницы
% insidebox, % ставим буквы в квадратики
% separateanswersheet, % добавляем бланк ответов
% nowatermark, % отсутствие надписи "Черновик"
% indivanswers,  % показываем верные ответы
% answers,
% lang=RU, % локализация слов "нет верных ответов", "вопрос" и тд
% completemulti % добавлять "нет правильного ответа" во всех вопросах множественного выбора
% ]{automultiplechoice}


\usepackage[colorlinks, hyperindex, unicode, breaklinks]{hyperref} % гиперссылки в pdf

\usepackage{amssymb}
\usepackage{amsmath}
\usepackage{amsthm}
\usepackage{epsfig}
\usepackage{bm}
\usepackage{color}

\usepackage{multicol}
\usepackage{multirow} % Слияние строк в таблице

\usepackage{textcomp}  % Чтобы в формулах можно было русские буквы писать через \text{}

%\usepackage{embedfile} % Чтобы код LaTeXа включился как приложение в PDF-файл

\usepackage{subfigure} % для создания нескольких рисунков внутри одного

\usepackage{tikz, pgfplots} % язык для рисования графики из latex'a
\usetikzlibrary{trees} % прибамбас в нем для рисовки деревьев
\usetikzlibrary{arrows} % прибамбас в нем для рисовки стрелочек подлиннее
\usepackage{tikz-qtree} % прибамбас в нем для рисовки деревьев

\usepackage{enumitem} % развернутые списки

% свешиваем пунктуацию (т.е. знаки пунктуации могут вылезать за правую границу текста, при этом текст выглядит ровнее)
\usepackage{microtype}

% более красивые таблицы
\usepackage{booktabs}
% заповеди из докупентации:
% 1. Не используйте вертикальные линни
% 2. Не используйте двойные линии
% 3. Единицы измерения - в шапку таблицы
% 4. Не сокращайте .1 вместо 0.1
% 5. Повторяющееся значение повторяйте, а не говорите "то же"

\usepackage{epigraph}

\usepackage{titleps} % заголовки страниц




% по поводу заголовков разделов в колонтитулах
% https://tex.stackexchange.com/questions/236705
% поэтому выбрали titleps вместо fancyhdr

\newpagestyle{mypage}{%
  \headrule
  \sethead{\sectiontitle}{}{\subsectiontitle}
  \setfoot{}{}{\thepage}
}
\settitlemarks{section,subsection,subsubsection} % !!!!!!no space after comma!!!!!!
\pagestyle{mypage}





\DeclareMathOperator{\Lin}{\mathrm{Lin}}
\DeclareMathOperator{\Linp}{\Lin^{\perp}}
\DeclareMathOperator*\plim{plim}
\DeclareMathOperator{\grad}{grad}
\DeclareMathOperator{\card}{card}
\DeclareMathOperator{\sgn}{sign}
\DeclareMathOperator{\sign}{sign}

\DeclareMathOperator*{\argmin}{arg\,min}
\DeclareMathOperator*{\argmax}{arg\,max}
\DeclareMathOperator*{\amn}{arg\,min}
\DeclareMathOperator*{\amx}{arg\,max}
\DeclareMathOperator{\cov}{Cov}
\DeclareMathOperator{\Var}{Var}
\DeclareMathOperator{\Cov}{Cov}
\DeclareMathOperator{\Corr}{Corr}
\DeclareMathOperator{\pCorr}{pCorr}
\DeclareMathOperator{\E}{\mathbb{E}}
\let\P\relax
\DeclareMathOperator{\P}{\mathbb{P}}


\newcommand{\cN}{\mathcal{N}}
\newcommand{\cU}{\mathcal{U}}
\newcommand{\cBinom}{\mathcal{Binom}}
\newcommand{\cBin}{\cBinom}
\newcommand{\cPois}{\mathcal{Pois}}
\newcommand{\cBeta}{\mathcal{Beta}}
\newcommand{\cGamma}{\mathcal{Gamma}}

\newcommand \R{\mathbb{R}}
\newcommand \N{\mathbb{N}}
\newcommand \Z{\mathbb{Z}}





\newcommand{\dx}[1]{\,\mathrm{d}#1} % для интеграла: маленький отступ и прямая d
\newcommand{\ind}[1]{\mathbbm{1}_{\{#1\}}} % Индикатор события
%\renewcommand{\to}{\rightarrow}
\newcommand{\eqdef}{\mathrel{\stackrel{\rm def}=}}
\newcommand{\iid}{\mathrel{\stackrel{\rm i.\,i.\,d.}\sim}}
\newcommand{\const}{\mathrm{const}}


% вместо горизонтальной делаем косую черточку в нестрогих неравенствах
\renewcommand{\le}{\leqslant}
\renewcommand{\ge}{\geqslant}
\renewcommand{\leq}{\leqslant}
\renewcommand{\geq}{\geqslant}



\AddEnumerateCounter{\asbuk}{\russian@alph}{щ} % для списков с русскими буквами
\setlist[enumerate, 2]{label=\asbuk*),ref=\asbuk*}




% делаем короче интервал в списках
\setlength{\itemsep}{0pt}
\setlength{\parskip}{0pt}
\setlength{\parsep}{0pt}

% \newenvironment{problem}{}{}
% тут перещёлкиваем комментарий, чтобы убрать или показать решения:
% \newenvironment{sol}{}{} % with solutions
% \excludecomment{sol} % without solutions



\unitlength=0.6mm

\title{Подборка экзаменов по теории вероятностей. \\Факультет экономики, НИУ ВШЭ}
\date{\today}
\author{Коллектив кафедры \\
математической экономики и эконометрики,
талантливые студенты,\\
 фольклор}


%%%%%%%%%%%%%%%%%% вставки
%%%%%%%%%%%%%%%%%%%%%%%%%%%%%%%%%%%%%%% Списки без уродских отступов
\newenvironment{enumerate*}{
\begin{enumerate}
  \setlength{\itemsep}{0pt}
  \setlength{\parskip}{0pt}
  \setlength{\parsep}{0pt}
}{\end{enumerate}}

\newenvironment{itemize*}{
\begin{itemize}
  \setlength{\itemsep}{0pt}
  \setlength{\parskip}{0pt}
  \setlength{\parsep}{0pt}
}{\end{itemize}}

\abovedisplayskip=0mm
\abovedisplayshortskip=0mm
\belowdisplayskip=0mm
\belowdisplayshortskip=0mm


% https://tex.stackexchange.com/questions/241343
% https://tex.stackexchange.com/questions/168480
\emergencystretch 5em % разрешаем дополнительные пробелы для упаковки параграфа до правой границы

% why this does not work?
\begin{hyphenrules}{russian}
\hyphenation{прав-до-по-до-бия диаг-нос-ти-ро-ван-ным экспо-нен-циаль-ному}
\end{hyphenrules}


% \setcounter{secnumdepth}{0} % убираем нумерацию секций, подсекций и т.д.


%%%%%%%%%%%
% блок для тестов
%%%%%%%%%%%
% [1][3] 1 = one argument, 3 = value if missing
% эта магия создаёт окружение answerlist
% именно в окружении answerlist записаны варианты ответов в подключаемых exerciseXX
% просто \begin{answerlist} сделает ответы в три столбца
% если ответы длинные, то надо в них руками сделать
% \begin{answerlist}[1] чтобы они шли в один столбец
\newenvironment{answerlist}[1][3]{
\begin{multicols}{#1}
\begin{enumerate}[label=\fbox{\emph{\Alph*}},ref=\emph{\alph*}]
}
{
\end{enumerate}
\end{multicols}
}


\excludecomment{solution} % without solutions

\theoremstyle{definition}

% опция [subsection] для сброса счётчика вопросов после каждой subsection
\newtheorem{question}{Вопрос}[subsection]

% чтобы номер вопроса был без номера секции:
\renewcommand{\thequestion}{\arabic{question}}
% конец блока для тестов
%%%%%%%%%%%%









\begin{document}

Ответы: AECBA BBCEB BBCCA BBCCC ABCBA EA?AC (? = был бы засчитан любой вариант ответа)

\begin{enumerate}
	
	%Вопрос №1
	\item  
	Из следствия неравенства Маркова: $\P(|X|>t)\leq\frac{\E|X|}{t}$, если $\E|X|<\infty$. 
	При этом случайная величина $X^2$ принимает только положительные значения.
	Соответственно, можем применить в этом задании формулу, эквивалентную формуле из следствия неравенства Маркова:
	
	\[
	\P(X^2\geq t)\leq \frac{\E(X^2)}{t}
	\]
	
	Получить $\E(X^2)$ можно из формулы дисперсии $X$:
	
	\[
	\Var(X)=\E(X^2)-[\E(X)]^2
	\]
	
	\[
	\E(X^2)=\Var(X)+[\E(X)]^2
	\]
	
	Теперь можем получить верхнюю границу для искомой вероятности:
	
	\[
	\P(X^2\geq 100)\leq \frac{\E(X^2)}{100} = 0.1
	\]
	
	Ответ: A [0;0.1] (нижнюю границу точнее установить невозможно, поэтому берём 0).
	
	%Вопрос №2
	\item
	
	Аналогично первой задаче $\E[\xi^2]$ получаем из формулы для дисперсии $X$:
	
	\[
	\E(\xi^2)=\Var(\xi)+[\E(\xi)^2]
	\]
	
	При этом известно, что для распределения Пуассона: $\E(\xi)=\Var(\xi)=\lambda$
	
	Тогда:
	
	\[
	\E(\xi^2)=\lambda+\lambda^2=\lambda(\lambda+1)
	\]
	
	Ответ: E
	
	
	%Вопрос №3
	\item 
	
	Известно, что
	
	\[
	\Corr(X+Y,Y)=\frac{\Cov(X+Y,Y)}{\sqrt{\Var(X+Y)\Var(Y)}}
	\]
	
	Найдём неизвестную по условию ковариацию $X+Y$ и $Y$ (помним, что $\Cov(Y,Y)=\Var(Y))$):
	
	\[
	\Cov(X+Y,Y)=\Cov(X,Y)+\Cov(Y,Y)=-3+9=6
	\]
	
	Теперь найдём дисперсию $X+Y$:
	
	\[
	\Var(X+Y)=\Var(X)+\Var(Y)+2\Сov(X,Y)=4+9+2\cdot(-3)=7
	\]
	
	Подставляем найденные значения в формулу корреляции случайных величин и получаем ответ:
	
	\[
	\Сorr(X+Y,Y)=\frac{6}{\sqrt{7\cdot9}}=\frac{2}{\sqrt{7}}
	\]
	
	Ответ: C
	
	
	
	%Вопрос №4
	\item 
	
	У случайной величины со стандартным нормальным распределением $\sigma=1$ и $\mu=0$. 
	При этом для любой случайной величины с нормальным распределением характерна следующая функция плотности:
	
	\[
	f(x)=\frac{1}{\sqrt{2\pi}\sigma} e^{-\frac{(x-\mu)^2}{2\sigma^2}}
	\]
	
	Подставляя $\sigma=1$ и $\mu=0$ в интеграл, получаем:
	
	\[
	\int\limits_{-1}^2 \frac{1}{\sqrt{2\pi}} e^{-\frac{x^2}{2}}\, dx
	\]
	
	Ответ: B
	
	
	
	%Вопрос 5
	\item 
	
	Поскольку двумерная случайная величина распределена равномерно в треугольнике, то значение её функции плотности в любой точке внутри треугольника равна:
	
	\[
	f(x,y)=\frac{1}{S}
	\]
	где $S$ - площадь треугольника
	
	Из уравнений, задающих стороны треугольника получаем, что они пересекают оси координат в точках $(0,0)$, $(0,4)$ и $(2,0)$. 
	Площадь такого треугольника равна:
	
	\[
	S=\frac{1}{2} \cdot 4 \cdot 2=4 
	\]
	
	Тогда функция плотности равна $\frac{1}{4}=0,25$
	
	Ответ: A
	
	
	%Вопрос 6
	\item 
	
	По определению A, B и C независимы в совокупности, если: 
	\[
	\P(A\cap B\cap C)=\P(A)\P(B)\P(C)
	\] 
	при условии, что также выполняются следующие равенства:
	
	\[
	\P(A\cap B)=\P(A)\P(B)
	\]
	
	\[
	\P(A\cap C)=\P(A)\P(C)
	\]
	
	\[
	\P(B\cap C)=\P(B)\P(C)
	\]
	
	Про эти 3 равенства ничего не сказано в вариантах ответов, но они предполагаются, поэтому правильным является тот ответ, где указано:
	
	\[
	\P(ABC)=\P(A)\P(B)\P(C)
	\]
	
	Ответ: B
	
	%Вопрос 7
	\item 	
	
	Если построить график функции плотности для распределения $\xi$, то получим прямоугольник, на оси x от 0 до 4 расположена одна из сторон которого, а другая сторона равна $\frac{1}{4}$ (значение функции плотности на отрезке $[0;4]$). 
	
	При этом за пределами отрезка $[0,4]$ значение функции плотности $\xi$ равно 0.
	
	Тогда получаем ответ через площадь указанного прямоугольника (учитывая, что из отрезка $[3;6]$ только часть $[3;4]$ с длиной 1 попадает в отрезок $[0;4]$, где функция плотности не равна 0):
	
	\[
	\P(\xi \in [3;6])=\P(\xi \in [3;4])=1\cdot \frac{1}{4}=\frac{1}{4}
	\]
	
	Ответ: B
	
	%Вопрос 8
	\item 
	
	Случайная величина $X$ может принимать 11 значений. 
	Значит, вероятность каждого равна $\frac{1}{11}$ (т.к. сказано, что случайная величина принимает эти значения равновероятно). 
	Аналогично для случайной величины $Y$ вероятность каждого значения равна $\frac{1}{3}$.
	
	Далее для поиска искомой вероятности $\P(X+Y^2 = 2)$ проще всего перебрать все возможные сочетания подходящих значений $X$ и $Y$, а затем просто сложить вероятности всех таких случаев. 
	
	Подходящие значения:
	
	\[\left[
	\begin{aligned}
	&Y=-1, X=1\\
	&Y=0, X=2\\
	&Y=1, X=1
	\end{aligned}
	\right.
	\]
	
	При этом из условия известно, что случайные величины $X$ и $Y$ независимы, а значит, вероятность одного сочетания равна произведению вероятностей попадания каждой из случайных величин в соответствующее значение:
	
	\[
	\P(Y=0, X=2)=\frac{1}{11}\cdot \frac{1}{3}=\frac{1}{33}
	\]
	
	Всего таких случаев 3, при этом их вероятности равны. 
	Поэтому можем просто умножить найденную вероятность на 3, чтобы получить искомую:
	
	\[
	\P(X+Y^2 = 2) = \frac{1}{33}\cdot 3=\frac{1}{11}
	\]
	
	Ответ: C	
	
	%Вопрос 9
	\item 
	
	Как известно, $\frac{\pi}{3}=60^{\circ}$. 
	Тогда доля каждого сектора равна $\frac{60}{360}=\frac{1}{6}$. 
	Поскольку все точки круга равновероятны, то вероятность попадания в конкретный сектор равна доле этого сектора.
	
	Соответственно, вероятность попадания в красный сектор (он один) равна:
	
	\[
	\P(\text{красный сектор})=\frac{1}{6}
	\]
	
	Ответ: E
	
	%Вопрос 10
	\item 
	
	\[
	\P(A\cup B)=\P(A)+\P(B)-\P(A\cap B)
	\]
	
	Выражаем из этого тождества $\P(B)$:
	
	\[
	\P(B)=\P(A\cup B)-\P(A)+\P(A\cap B)
	\]
	
	Подставляем значения из условия и получаем:
	
	\[
	\P(B)=0.6-0.3+0.2=0.5
	\]
	
	Ответ: B
	
	%Вопрос 11
	\item 
	
	\[
	\Var(2X-Y+1)=\Var(2X-Y)=4\Var(X)+\Var(Y)-4)\Cov(X,Y)=4\cdot 4+9-4\cdot(-3)=37
	\]
	
	Ответ: B
	
	%Вопрос 12
	\item 
	
	Согласно ЗБЧ указанная величина сходится по вероятности к $\E(X^2)$. 
	Тогда получаем:
	
	\[
	\plim_{n\to\infty}\frac{X^2_{1}+X^2_{2}+\ldots+X^2_{n}}{n}=\E(X^2)=\Var(X)+\E^2(X)=1+0=1
	\]
	
	Ответ: B
	
	%Вопрос 13
	\item 
	
	\[
	f\Bigr(x|y=\frac{1}{2}\Bigl)=\frac{f(x,\frac{1}{2})}{f_{Y}(\frac{1}{2})}
	\]
	
	В числителе справа стоит совместная функция плотности, в которую необходимо подставить значение $y=\frac{1}{2}$. 
	Найдём функцию плотности случайной величины Y из совместной функции плотности:
	
	\[
	f_{Y}(y)=\int\limits_{-\infty}^{\infty} f(x,y)\,dx=\int\limits_0^1 6xy^2\,dx=3y^2
	\] 
	
	Подставляем в формулу найденную ранее и находим ответ:
	
	\[
	f\Bigr(x|y=\frac{1}{2}\Bigl)=\frac{6x^2\cdot \frac{1}{4}}{3\cdot \frac{1}{4}}=2x
	\]
	
	Ответ: C
	
	%Вопрос 14
	\item 
	
	В условиях задачи согласно ЦПТ $\overline{X}$ сходится к нормальному распределению с параметрами $\E(X_{i})$ и $\frac{\Var(X_{i})}{n}$. 
	То есть асимптотически выполняется следующее:
	
	\[
	\overline{X}\sim \N \Bigr(4, \frac{100}{100}\Bigl)
	\]
	
	Тогда с помощью стандартизации случайной величины получаем:
	
	\[
	\P(\overline{X}\leqslant 5)=\P(\frac{\overline{X}-4}{\sqrt{1}}\leqslant \frac{5-4}{1})=\P(Z\leqslant 1)=0.8413
	\]
	
	Ответ: C
	
	%Вопрос 15
	\item 
	
	\begin{gather*}
	\Cov(X+2Y, 2X+3)=\Cov(X+2Y, 2X)=\Cov(X,2X)+\Cov(2Y,2X)=\\=2\Var(X)+4\Cov(X,Y)=2\cdot4+4\cdot(-3)=-4
	\end{gather*}
	
	Ответ: A
	
	%Вопрос 16
	\item 
	
	\[
	\E((X-1)Y)=\E(XY-Y)=\E(XY)-\E(Y)=\Cov(X,Y)+\E(X)\E(Y)-\E(Y)=-3+(-1)\cdot2-2=-7
	\]
	
	Ответ: B
	
	%Вопрос 17
	\item 
	
	Случайная величина $X_{i}$ имеет распределение Бернулли:
	
	\[
	X_{i}=\begin{cases}
	1, &\frac{1}{6}\\
	0, &\frac{5}{6}\\
	\end{cases}
	\]
	
	Чтобы определить распределение $X_{1}$ при условии $X_{1}+X_{2}=1$ необходимо сначала вычислить вероятность этого условия. 
	Проще всего это сделать, если определить какие значения случайных величин подходят для выполнения условий, а затем сложить вероятности подходящих случаев. 
	Для начала найдём такие случаи:
	
	\[
	\begin{cases}
	X_{1}=1\, \text{и}\, X_{2}=0, &p=\frac{1}{6}\cdot \frac{5}{6}=\frac{5}{36}\\
	X_{1}=0\, \text{и}\, X_{2}=1, &p=\frac{5}{6}\cdot \frac{1}{6}=\frac{5}{36}\\
	\end{cases}
	\]
	
	Тогда вероятность события из условия равна:
	
	\[
	\P(X_{1}+X_{2})=\frac{5}{36}\cdot 2
	\]
	
	Для определения условного распределения $X_{1}$ осталось только определить условные вероятности значений этой случайной величины:
	
	\[
	\P(X_{1}=0|X_{1}+X_{2}=1)=\frac{\P(X_{1}=0 \cap X_{1}+X_{2}=1)}{\P(X_{1}+X_{2}=1)}=\frac{\frac{5}{36}}{\frac{5}{36}\cdot 2}=\frac{1}{2}
	\]
	
	Аналогично для значения 1:
	
	\[
	\P(X_{1}=1|X_{1}+X_{2}=1)=\frac{1}{2}
	\]
	
	Поскольку получили вероятности по $\frac{1}{2}$ для каждого из двух значений случайной величины, то можем утверждать, что условное распределение $X_1$ при заданном условии совпадает с распределением Бернулли с $p=\frac{1}{2}$.
	
	Ответ: B
	
	%Вопрос 18
	\item 
	
	Случайная величина $X+Y$ имеет нормальное распределение с параметрами $\E(X+Y)=\E(X)+\E(Y)=2+1=3$ и $\Var(X+Y)=\Var(X)+\Var(Y)=3+4=7$ (дисперсия суммы равна сумме дисперсий, поскольку случайные величины независимые).
	Стандартизируя эту случайную величину, получаем ответ:
	
	\[
	\P(X+Y<3)=\P\Bigr(\frac{X+Y-3}{\sqrt{7}}<\frac{3-3}{\sqrt{7}}\Bigl)=\P(Z<0)=0.5
	\]
	
	Ответ: C
	
	%Вопрос 19
	\item 
	
	В задаче необходимо найти условную вероятность по следующей формуле:
	
	\[
	\P(\text{"честный кубик"|"6"})=\frac{\P(\text{"честный кубик"} \cap \text{"6"})}{\P(\text{"6"})}
	\]
	
	Значение числителя находим через перемножение вероятностей для каждого из множеств:
	
	\[
	\P(\text{"честный кубик"} \cap \text{"6"})=\frac{3}{5}\cdot \frac{1}{6}
	\]
	
	По формуле полной вероятности найдём значение знаменателя:
	
	\[
	\P("6")=\frac{1}{5}\cdot \frac{1}{6}+\frac{1}{5}\cdot \frac{1}{6}+\frac{1}{5}\cdot \frac{1}{6}+\frac{1}{5}\cdot \frac{1}{2}+\frac{1}{5}\cdot \frac{1}{10}=\frac{11}{50}
	\]
	
	Подставляем полученные значения в первую формулу и получаем ответ:
	
	\[
	\P(\text{"честный кубик"|"6"})=\frac{\frac{3}{5}\cdot \frac{1}{6}}{\frac{11}{50}}=\frac{5}{11}
	\]
	
	Ответ: C
	
	%Вопрос 20
	\item 
	
	На первом шаге методом исключения отбрасываем вариант D, поскольку дисперсия не может быть отрицательной, а в этой матрице на главной диагонали есть отрицательное число. 
	Затем отбрасываем матрицу E, поскольку она не является симметричной (на побочной диагонали стоят разные числа).
	
	Далее из оставшихся матриц выбираем нужную после подсчёта определителя (у ковариационной матрицы определитель должен быть положительным):
	
	\[
	\det A=1\cdot 1-2\cdot 2=-3<0
	\]
	
	\[
	\det B=1\cdot 9-4\cdot 4=-7<0
	\]
	
	\[
	\det C=9\cdot 6-7\cdot 7=5>0
	\]
	
	Ответ: C
	
	%Вопрос 21
	\item 
	
	\[
	\E(aX+(1-a)Y)=a\E(X)+(1-a)\E(Y)
	\]
	
	Подставляем значения из условия, чтобы получить уравнение с $a$: 
	
	\[
	-a+2(1-a)=0
	\]
	
	\[
	2-3a=0
	\]
	
	\[
	a=\frac{2}{3}
	\]
	
	Ответ: A
	
	%Вопрос 22
	\item 
	
	Формула вероятности для случайной величины с биномиальным распределением имеет следующий вид:
	
	\[
	\P(\xi=k)=p^k (1-p)^{n-k} при k\in{0,1,2,\ldots}
	\]
	
	Тогда искомая вероятность равна:
	
	\[
	\P(\xi=0)=\Bigr(\frac{3}{4}\Bigl)^0 \Bigr(\frac{1}{4}\Bigl)^{2-0}=\Bigr(\frac{1}{4}\Bigl)^2=\frac{1}{16}
	\]
	
	Ответ: B
	
	%Вопрос 23
	\item 
	
	Пусть X - количество сбоев системы за сутки. 
	Для случайной величины с распределением Пуассона верно следующее:
	
	\[
	\P(X=k)=\lambda^k\cdot \frac{\exp^{-\lambda}}{k!} \, \text{при} \, k\in{0,1,2,\ldots}
	\]
	
	Тогда искомая вероятность равна:
	
	\[
	\P(X\geqslant 1)=1-\P(X=0)=1-e^{-4}
	\]
	
	Ответ: C
	
	%Вопрос 24
	\item 
	
	\[
	\xi=\begin{cases}
	1 ,\, p\\
	0, \, 1-p\\
	\end{cases}
	\]
	
	Из формулы дисперсии следует:
	
	\[
	\E(\xi^2)=\Var(\xi)+\E(\xi)^2
	\]
	
	При этом, как известно, у случайных величин с распределением Бернулли:
	
	\[
	\E(\xi)=p, \Var(\xi)=p(1-p)
	\]
	
	\[
	\E(\xi^2)=p(1-p)+p^2=p
	\]
	
	Ответ: B
	
	%Вопрос 25
	\item 
	
	\[
	\E(\xi)=\frac{1}{\lambda}, \Var(\xi)=\frac{1}{\lambda^2}
	\]
	
	\[
	\E(\xi^2)=\frac{1}{\lambda^2}+\frac{1}{\lambda^2}=\frac{2}{\lambda^2}
	\]
	
	Ответ: A
	
	%Вопрос 26
	\item 
	
	Методом исключения отбираем правильный вариант ответа. 
	
	События A и B не образуют полную группу событий, поскольку два сектора, которые они характеризуют, не покрывают всю площадь круга, в которую может попасть Вася.
	Вероятности этих событий (каждого) равны $\frac{2}{6}$. 
	Соответственно, неверны пункты B и D. 
	Наконец, эти события не являются независимыми, т.к., например, попадание в красный сектор (событие A) означает невозможность попадания в синий сектор (событие B).
	Остаётся вариант про несовместность этих событий, который и является верным.
	
	Ответ: E
	
	%Вопрос 27
	\item 
	
	\[
	\E(XY)=\int\limits_{0}^{1}\int\limits_{0}^{1} xy\cdot 6xy^2\,dxdy=\int\limits_{0}^{1} 2x^3y^3\,dy=\frac{2y^4}{4}=\frac{1}{2}
	\]
	
	Ответ: A	
	
	%Вопрос 28
	\item 
	
	\begin{gather*}
	\Var(aX+(1-a)Y)=a^2 \Var(X)+(1-a)^2 \Var(Y)+2\Cov(X,Y)\cdot a(1-a)=\\=4a^2+9(1-a)^2-6a(1-a)=4a^2+9-18a+9a^2-6a+6a^2=19a^2-24a+9
	\end{gather*}
	
	Получаем уравнение, задающее параболу ветвями вверх. 
	Его необходимо минимизировать, для этого достаточно найти координату вершины этой параболы: она гарантированно будет минимумом.
	
	\[
	a=-\frac{-24}{2\cdot 19}=\frac{12}{19}
	\] 
	
	Ответ: правильного ответа среди вариантов не было (в таком случае был бы засчитан любой ответ)
	
	%Вопрос 29
	\item 
	
	Чтобы можно было ориентироваться в финальной формуле, запишем сначала всё, что известно из условия в более удобном формате:
	
	\[
	\P(\text{без багажа})=\frac{1}{4}
	\]
	
	\[
	\P(\text{с рюкзаком|без багажа})=\frac{1}{2}
	\]
	
	\[
	\P(\text{с рюкзаком|с багажом})=\frac{55}{150}
	\]
	
	Теперь, используя формулу полной вероятности, получим искомый ответ:
	
	\begin{gather*}
	\P(\text{без рюкзака})=\P(\text{без рюкзака}|\text{без багажа})\cdot \P(\text{без багажа})+\P(\text{без рюкзака}|\text{с багажом})\cdot \P(\text{с багажом})=\\=\frac{1}{2} \cdot \frac{1}{4}+\frac{95}{150}\cdot \frac{3}{4}=0.6
	\end{gather*}
	
	Ответ: A
	
	%Вопрос 30
	\item 
	
	Искомая вероятность похожа на неравенство Чебышёва, однако внутри скобки знак "меньше", а в неравенстве Чебышёва знак "больше". 
	
	Запишем неравенство Чебышёва, как оно могло бы выглядеть в условиях этой задачи:
	
	\[
	\P(|X-2|\geqslant 10)\leqslant \frac{\Var(X)}{100}
	\]
	
	Теперь можем вычислить искомую вероятность, используя это неравенство:
	
	\[
	\P(|2-X|\leqslant 10)=\P(|X-2|\leqslant 10)\geqslant 1-\P(|X-2|\geqslant 10)=1-\frac{\Var(X)}{100}=1-\frac{6}{100}=0.94
	\]
	
	Ответ: C
\end{enumerate}
\end{document}
